\documentclass{article}
\usepackage[utf8]{inputenc}

\usepackage[T2A]{fontenc}
\usepackage[utf8]{inputenc}
\usepackage[russian]{babel}

\usepackage{multienum}
\usepackage{geometry}
\usepackage{hyperref}

\geometry{
    left=1cm,right=1cm,
    top=2cm,bottom=2cm
}

\usepackage{graphicx}
\graphicspath{ {./images/} }

\usepackage{xcolor,mdframed}
\newenvironment{important}[1][]{%
   \begin{mdframed}[%
      backgroundcolor={red!15}, hidealllines=true,
      skipabove=0.7\baselineskip, skipbelow=0.7\baselineskip,
      splitbottomskip=2pt, splittopskip=4pt, #1]%
   \makebox[0pt]{% ignore the withd of !
      \smash{% ignor the height of !
         \fontsize{32pt}{32pt}\selectfont% make the ! bigger
         \hspace*{-19pt}% move ! to the left
         \raisebox{-2pt}{% move ! up a little
            {\color{red!70!black}\sffamily\bfseries !}% type the bold red !
         }%
      }%
   }%
}{\end{mdframed}}

\title{История}
\author{Лисид Лаконский}
\date{February 2023}

\newtheorem{definition}{Определение}

\begin{document}
\raggedright

\maketitle
\tableofcontents
\pagebreak

\section{Практическое занятие по истории №9, «россия в 1917 году: выбор путей общественного развития»}

\subsection{Социально-экономическое и политическое положение страны в конце 1916 – начале 1917 г}

\begin{important}
    Будьте осторожны. Текст, приведенный ниже, выглядит так, будто он из учебника по истории КПСС
\end{important}

 \textbf{Осенью 1916 г.} в России разразился общенациональный политический кризис, охвативший все стороны социально-экономической и политической жизни, все классы и партии. Война подорвала производительные силы страны, втянула ее в длительную полосу экономической разрухи. В \textbf{конце 1916 – начале 1917 г.} сокращается производство в ряде ведущих отраслей петроградской промышленности, закрываются десятки заводов и фабрик. В Москве из-за недостатка топлива закрылись 14 промышленных предприятий, под угрозой полной или частичной остановки находилось еще 20 предприятий. В Екатеринославском промышленном районе из 130 предприятий, объединяемых обществом заводчиков и фабрикантов, большинство работало в половину своих мощностей. В Брянске на неопределенное время был закрыт Бежицкий паровозостроительный завод; под угрозой закрытия находились Тульские оружейные и Сормовские металлообрабатывающие заводы. В результате десятки тысяч пролетариев лишались возможности трудиться, пополняли армию безработных.

 \hfill

Материальное положение российского пролетариата, и прежде всего питерских и московских рабочих, в \textbf{конце 1916 – начале 1917 г.} стало совершенно невыносимым. Нищенский жизненный уровень абсолютного большинства рабочих находился в остром противоречии как с их материальными потребностями, так и с классовой сознательностью и организованностью. Революционный натиск на самодержавие был неизбежен. Резкое обострение «выше обычного» нужды и бедствий пролетариев и их семей составляло материальную сторону (наиболее близкую и понятную им) их нежелания жить «по-старому», осознания необходимости ликвидации монархического строя.

\hfill

В \textbf{осенне-зимние месяцы 1916-1917 гг.} резко повышается революционная активность российского рабочего класса. Несмотря на постоянные репрессии царизма, массовые мобилизации в армию, обновление состава, пролетариат все более наращивал мощь своих стачечных ударов по самодержавию. С новой силой забастовочная борьба разгорается в \textbf{октябре 1916 г.}: в стране бастовало 187 475 рабочих, из них 135 310 приняли участие в политических стачках. Следующая волна стачек (после временного затишья в ноябре – декабре 1916 г.) поднимается в январе 1917 г. В феврале 1917 г. они перерастают в Февральскую революцию. Причем если в период с января по июнь 1916 г. в политических выступлениях принимало участие 22,6\%, а в экономических – 77,4\% всех стачечников, в период с июля по сентябрь 1916 г.- соответственно 15,3 и 84,7\%, то в октябре – декабре в политических забастовках приняло участие уже 58\%, а в экономических – 42\% всех стачечников; в январе 1917 г.- соответственно 65 и 35\%.

\pagebreak
\subsection{Падение монархии в феврале 1917 года. Временное правительство и Петросовет: проблема взаимоотношений}

Февральская революция 1917 года в России — \textbf{массовые антиправительственные выступления петроградских рабочих и солдат петроградского гарнизона, приведшие к свержению российской монархии и созданию Временного правительства}, сосредоточившего в своих руках всю законодательную и исполнительную власть в России. Революционные события охватили период конца февраля — начала марта 1917 года (по юлианскому календарю, действовавшему в то время в России).

\hfill

Началась как стихийный порыв народных масс в условиях острого политического кризиса власти, резкого недовольства либерально-буржуазных кругов единоличной политикой царя, «брожения» среди многотысячного столичного гарнизона, присоединившегося к революционным массам. \textbf{27 февраля (12 марта) 1917 года} всеобщая забастовка переросла в вооружённое восстание; войска, перешедшие на сторону восставших, заняли важнейшие пункты города, правительственные здания. Разрозненные и немногочисленные силы, сохранившие верность царскому правительству, не смогли справиться с охватившим столицу хаосом, а несколько частей, снятых с фронта, не смогли пробиться к городу.

\hfill

Непосредственным результатом Февральской революции стало \textbf{отречение от престола Николая II, прекращение правления династии Романовых}. Всю власть в стране взяло \textbf{Временное правительство под председательством князя Георгия Львова}, тесно связанное с буржуазными общественными организациями, возникшими в годы войны (Всероссийский земский союз, Городской союз, Центральный военно-промышленный комитет). Временное правительство объявило амнистию политическим заключённым, гражданские свободы, замену полиции «народной милицией», реформу местного самоуправления.

\hfill

Практически одновременно революционно-демократическими силами был сформирован параллельный орган власти — \textbf{Петроградский Совет} — что привело к ситуации, известной как двоевластие.

\hfill

1 (14) марта 1917 года новая власть была установлена в Москве, в течение марта — по всей стране.

\hfill

В конце 1917 года \textbf{Временное правительство было свергнуто в ходе Октябрьского вооружённого восстания в Петрограде, и к политической власти пришла большевистская партия}. Две революции ознаменовали кардинальные перемены в государственном устройстве России: Февральская революция привела к свержению монархии, Октябрьская — к установлению советской власти — совершенно новой формы правления.

\subsubsection{Двоевластие: временное правительство и петросовет}

Двоевластие — сосуществование параллельных систем власти и управления в России после Февральской революции в феврале-сентябре 1917 года:

\begin{enumerate}
    \item системы, связанные с официальной властью — органами Временного правительства, регионального и городского управления, политическими и сословно-профессиональными организациями образованных и имущих слоёв населения,
    \item системы, возникшей на базе Советов, их общегосударственных и региональных объединений и включавшей те политические организации, которые были либо представлены в Советах, либо ориентировались на них. В столице двоевластие проявилось в разделе власти между Петросоветом и Временным правительством, на местах — между Советами и комиссарами Временного правительства и комитетами общественных организаций.
\end{enumerate}

Больше информации по ссылке: \href{https://ru.wikipedia.org/wiki/%D0%94%D0%B2%D0%BE%D0%B5%D0%B2%D0%BB%D0%B0%D1%81%D1%82%D0%B8%D0%B5_%D0%B2_%D0%A0%D0%BE%D1%81%D1%81%D0%B8%D0%B8_1917_%D0%B3%D0%BE%D0%B4%D0%B0#%D0%92%D0%B7%D0%B0%D0%B8%D0%BC%D0%BE%D0%B4%D0%B5%D0%B9%D1%81%D1%82%D0%B2%D0%B8%D0%B5%20%D0%B2%D0%BB%D0%B0%D1%81%D1%82%D0%BD%D1%8B%D1%85%20%D1%86%D0%B5%D0%BD%D1%82%D1%80%D0%BE%D0%B2}{Раздел «Взаимодействие властных центров», статья «Двоевластие в России 1917 года», Википедия}

\pagebreak
\subsection{Альтернативы политического развития России после Февраля}

\begin{important}
    Весь текст, приведенный ниже, является перепечаткой нагугленного реферата по истории
\end{important}

\subsubsection{Конституционные проекты Временного правительства}

Члены Временного правительства понимали, что \textbf{одной из причин революции является отсутствие конституционных преобразований}. Смена формы правления требовала реализации срочных конституционных реформ. Необходимо было юридически закрепить новую форму правления, разработать механизмы управления системой власти.

\hfill
К выработке проекта конституции в порядке подготовки к Учредительному собранию Особая комиссия приступила \textbf{11 октября 1917 г.} Было выделено 18 основных вопросов, по которым нужно было принять решения и продумать механизм их исполнения. Наиболее знаковым проектом Особой комиссии, раскрывающий суть альтернативы буржуазно-реформистского пути развития России, стал проект \textbf{«Об организации временной исполнительной власти при Учредительном собрании»}. Содержание проекта включало способы решения различных проблем в России:

\begin{enumerate}
    \item форма государственного правления - \textbf{президентская республика}. Был предложен внепарламентский метод избрания президента на Учредительном собрании. Председатель Совета Министров и министры являются ответственными за весь процесс государственного управления, отвечают перед Учредительным собранием. Таким образом, \textbf{правовой статус президента схож с правовым статусом монарха, который был закреплен в Основных государственных законах Российской империи 1906 г}
    \item в вопросе о форме государственного устройства основным являлся тезис кадетов о «единой и неделимой России». Федеративные отношения планировалось установить только с Финляндией.
\end{enumerate}

На заседании \textbf{20 октября 1917 г.} были приняты \textbf{«Тезисы по вопросу о верхней палате»}, которые определяли основные положения о структуре парламентской власти в стране. Предполагалось создание двухпалатного парламента: нижняя палаты должна быть представлена выборным населением, членами верхней палаты должны были являться представители автономных учреждений областей. «Кроме того, представлялось желательным, чтобы в состав верхней палаты вошли: представители торговли и промышленности, кооперативов, профессиональных союзов, академий, учебных обществ и высших учебных заведений».

\hfill

\textbf{Однако идеи Временного правительства реализовать не удалось}. Ухудшение социально-экономической ситуации в связи с войной и революцией, формирование параллельного органа власти революционных масс (Петросовет) являлись препятствием к осуществлению реформистских преобразований буржуазного характера.

\hfill

В условиях обострявшегося кризиса власти Временное правительство выступает с инициативой об \textbf{усилении уголовной ответственности за насильственное посягательство на изменение существующего строя в России}. Данный указ \textbf{дискредитировал Временное правительство в глазах народа: революционные массы расценили это действие как стремление Временного правительства отказаться от реформ}. После направления Временным правительством воинских отрядов для принудительной реквизиции хлеба в деревню авторитет Временного правительства окончательно пал. Итогом стали Октябрьская большевистская революция и разгон Учредительного собрания. \textbf{Таким образом, конституционные проекты Временного правительства, являясь соединением основных законов Российской империи от 23 апреля 1906 года}, представляли идеи западноевропейского конституционализма и являлись буржуазно-демократической альтернативой развития России после февральских событий 1917 г.

\subsubsection{Альтернативы развития России до победы большевиков}

Другими альтернативами развития России после Февральской революции 1917 г. можно обозначить следующие:

\begin{enumerate}
    \item \textbf{Союз Керенского и правых социалистов}. К октябрю 1917 г. Керенский потерял доверие рабочих и солдат, и партии эсеров (эсеры на III съезде не избрали Керенского в состав нового ЦК). Однако решения вопросов о мире и земле раньше распространения идей большевистских декретов могли лишить Ленина возможности выдвинуться на съезде на первый план и возглавить правительство. Правые социалисты сумели бы добиться создания такого коалиционного правительства, которое довело бы страну до Учредительного собрания.
    \item \textbf{Военная диктатура генерала Корнилова}. Успех мятежа генерала Корнилова мог привести к реставрации консервативно-буржуазных порядков и в дальнейшем к формированию конституционной монархии. Однако генерал Корнилов как представитель и проводник буржуазно-помещичьей реставрации не имел поддержки широких слоев населения, не сумел найти опору в солдатских рядах.
    \item \textbf{Альтернатива многопартийного советского правительства}.
\end{enumerate}

\textbf{Альтернатива создания советского правительства как правительства разных политических сил являлась реальной до лета 1918 г.} Понимание правыми социалистами разногласий с большевиками и невозможность договориться привела к созданию Комуч. Вооруженное выступление против власти Советов демонстрировала раскол социалистического лагеря в России.

\hfill

Таким образом, альтернатива многопартийного советского правительства потерпела крах: уже в июне 1918 г. большевики исключили правых социалистов из состава ВЦИК Советов. Однако еще сохранялась возможность формирования двухпартийной политической системы (вплоть до июля 1918 г. ). В состав крестьянских советов в большинстве входили левые эсеры, численность которых возрастала (их популярность в деревне была в разы выше большевистской). На советских съездах возрастало количество крестьянских советов, соответственно, возрастало и количество делегатов от левых эсеров. Такими темпами реальной становилась возможность формирования левоэсеровско-большевистского правительства.

\hfill

Таким образом, \textbf{новое советское государство имело бы двухпартийную политическую систему}. В дальнейшем сохранялась возможность пути демократического развития России. Левые эсеры упустили возможность остаться у власти. Опираясь на опыт большевиков в октябрьских событиях, \textbf{левые эсеры отказались от тактики ожидания и в июле 1918 г. ими был организован мятеж}. Естественно, после мятежа левые эсеры вошли в ряды нелегальной оппозиции, борьба с большевиками в рамках Советов прекратилась. Во избежание повторения ситуации \textbf{большевиками был разработан новый избирательный закон, согласно которому, соотношение голосов рабочего и крестьянина составляло 1 к 5}. Россия стала развиваться по пути, предложенному партией большевиков.

\hfill

Таким образом, после Февральской революции 1917 г. возникло несколько альтернативных путей развития российского государства: \textbf{буржуазно-демократический путь, революционная демократия, военная диктатура}. На разных этапах развития событий с февраля 1917 г. по июль 1918 г. перспективы выбора разных альтернатив менялись. В итоге к власти пришли большевики, ликвидировав всех политических соперников, обозначили исторический путь России на несколько десятилетий

\subsubsection{Заключения}

Подводя итог проделанной работе следует обозначить следующие выводы:

\begin{enumerate}
    \item На протяжении периода февраль 1917 г. - июль 1918 г. \textbf{Россия находилась перед историческим выбором будущего государственного устройства}. Выбор этот был представлен несколькими различными по характеру и политической основой альтернативами развития.
    \item Медлительность и нерешительность в решении вопросов о войне и мире в условиях нарастающей революции и мировой войны привели к \textbf{дискредитации Временного правительства и краху альтернативы буржуазно-демократического пути развития страны}.
    \item Политическая недальновидность Керенского в вопросе об отношениях с левыми эсерами стала причиной \textbf{краха альтернативы союза социалистов и созыва Учредительного собрания}.
    \item Отсутствие у генерала Корнилова социалистического прикрытия Керенского \textbf{привело шансы военно-диктаторской альтернативы к краху}.
    \item Поражение варианта военной диктатуры генерала Корнилова оставило на политической арене большевиков, меньшевиков и эсеров. Объединение эсеров и меньшевиков в сентябре имело реальные шансы соперничать с большевиками за симпатии народа. \textbf{Однако эсеры и меньшевики не проявили активности в решении вопросов о мире и земле. В процессе работы Демократического совещания и Предпарламента альтернатива противостояния большевиков с эсеро-меньшевистской оппозицией окончательно рухнула}.
\end{enumerate}

\pagebreak
\subsection{Причины обострения политической ситуации в послефевральский период}

\pagebreak
\subsection{Экономическая политика Временного правительства и ее результаты}

\textbf{Реализация экономической программы Временного правительства встретила серьезные трудности}. Сказался фактор войны, отсутствие твердой воли в выполнении ряда непопулярных решений (госмонополии, налог на сверхприбыль и др.), сопротивление предпринимателей. Курс на усиление государственного регулирования экономики реализовывался в двух направлениях:

\begin{enumerate}
    \item введение ряда госмонополий (на хлеб, уголь, сахар)
    \item создание экономического центра, вырабатывающего единый план
\end{enumerate}

Однако ни то, ни другое \textbf{не дало серьезных практических результатов}. \textbf{25 марта 1917 г. был издан закон о хлебной монополии}, по которому все зерно, сверх необходимого для потребления, посевов и для корма скоту, должно было отчуждаться по твердым ценам в общегосударственный фонд для последующего перераспределения. Хлеба в стране было достаточно, так как собирались неплохие урожаи и почти прекратился экспорт хлеба в связи с войной. Валовой сбор зерна в 1917 г. составил 56 млн т, в 1916 г. — 57 млн т, товарный фонд — 23,4 млн т при потреблении 17,6 млн т. Во многих губерниях Сибири, Северного Кавказа, Украины имелись большие запасы хлеба от урожаев 1915 и 1916 гг. Хлебная монополия предполагала установление твердых цен как на хлеб, так и на товары, необходимые крестьянам. \textbf{Владельцы хлеба были недовольны низкими закупочными ценами, которые были значительно ниже рыночных}.

\hfill

\textbf{5 мая 1917 г. Временное правительство образовало министерство продовольствия}, имевшее широкие полномочия: учет запасов хлеба, их отчуждение и распределение, снабжение производителей семенами, сельхозорудиями, а населения продовольствием, регулирование рыночного оборота, установление цен. Под давлением держателей хлеба правительство в августе повысило заготовительные цены в два раза. Это позволило увеличить заготовки хлеба, но привело к его удорожанию в городах. Трудности снабжения усугублялись плохой работой транспорта, который не справлялся с перевозками. \textbf{Продолжала действовать карточная система, введенная в 1916 г. Нормы постоянно снижались}. Например, Москва и центральные губернии в сентябре 1917 г. получали лишь 12\% необходимого продовольствия. Иногда в день выдавалось всего по 100 граммов хлеба.

\hfill

Были предприняты \textbf{попытки централизованного учета сахара, кожи, железа, керосина, масла, бумаги и других товаров, но они дали слабые результаты}. В августе введена угольная монополия, в сентябре — сахарная.

\hfill

Государственное регулирование производства осуществлялось посредством \textbf{государственных заказов}. По таким заказам работала металлургическая промышленность, которая в централизованном порядке снабжалась сырьем и топливом и поставляла продукцию казне по фиксированным ценам. Таким же образом регулировалась нефтяная промышленность.

\hfill

Для выработки общей экономической стратегии Временное правительство в июне сформировало \textbf{Экономический совет}. На него возлагались две задачи:

\begin{enumerate}
    \item выработка плана и постепенное регулирование жизни страны в общегосударственных интересах
    \item осуществление экспертизы всех разрабатываемых хозяйственных мероприятий для обеспечения целостности проводимой экономической политики.
\end{enumerate}

Председателем Экономсовета стал министр торговли и промышленности С.Н. Прокопович. В работе Совета участвовали многие видные экономисты и хозяйственные деятели: П.И. Пальчинский, Н.Д. Кондратьев, П.Б. Струве, В.А. Базаров,

\hfill

Суть государственного регулирования экономики члены Совета видели в том, чтобы изымать в централизованный фонд производимую на предприятиях продукцию и перераспределять ее в соответствии с указаниями государства. Таким образом, \textbf{предполагалось в довольно широких масштабах использовать принуждение}. Оно проявлялось в государственном регулировании (а фактически в установлении) цен, заработной платы, в формировании потребления. Ставился вопрос о введении всеобщей трудовой повинности. Однако \textbf{все эти проекты до практического осуществления не дошли}.

\hfill

Регулирование экономики возлагалось на \textbf{Главный экономический комитет}, созданный одновременно с Экономсоветом. В проекте Положения о нем говорилось: «Экономический комитет, руководствуясь общими указаниями Экономического совета, разрабатывает и утверждает все планы массовых заготовлений, согласуя спрос с возможностью его удовлетворения». Его ведению подлежали утверждение цен на основные виды сырья и готовой продукции, регулирование заработной платы путем установления ее норм в основных отраслях, распределение заказов.

\hfill

При Экономическом комитете создавалось \textbf{Экономическое совещание с функциями центрального комитета снабжения}. Оно составляло государственный план снабжения армии и населения необходимыми продуктами и товарами и состояло из председателей всех специальных органов снабжения и военных комитетов снабжения, а также представителей общественных организаций.

\hfill

Таким образом, \textbf{планы Временного правительства были обширными, но из них мало что удалось осуществить}.

\pagebreak
\subsection{Экономические программы политических партий в 1917 году}

\subsubsection{Программа партии социалистов-революционеров (эсеров)}

Эсеры \textbf{являлись прямыми наследниками старого народничества, сущность которого составляла идея о возможности перехода России к социализму некапиталистическим путём}. В то же время эсеры были сторонниками демократического социализма, то есть хозяйственной и политической демократии, которая должна была выражаться через \textbf{представительство организованных производителей} (профсоюзы), организованных потребителей (кооперативные союзы) и организованных граждан (демократическое государство в лице парламента и органов самоуправления).

\hfill

Оригинальность эсеровского социализма заключалась в теории \textbf{«социализации земледелия»}. Эта теория составляла национальную особенность эсеровского демократического социализма и являлась вкладом в развитие мировой социалистической мысли. Исходная идея этой теории заключалась в том, что \textbf{социализм в России должен начать произрастать раньше всего в деревне}. Почвой для него, его предварительной стадией, должна была стать «социализация земли». 

\hfill

\textbf{«Социализация земли»} означала:

\begin{enumerate}
    \item \textbf{отмену частной собственности на землю, превращение её в общенародное достояние} без права купли-продажи, без преобразования её в государственную собственность, без национализации;
    \item \textbf{переход всей земли под руководство центральных и местных органов народного самоуправления}, начиная от демократически организованных сельских и городских общин и заканчивая областными и центральными учреждениями;
    \item \textbf{пользование землёй становится уравнительно-трудовым}, то есть обеспечивать потребительную норму на основании приложения собственного труда, единоличного или в товариществе.
\end{enumerate}

\subsubsection{Экономическая программа большевиков}

Февральская буржуазно-демократическая революция покончила с режимом самодержавия и всевластия крепостников-помещиков. В стране создалось двоевластие, переплетение двух диктатур.

\hfill

\textbf{Наряду с Временным правительством возникли Советы рабочих и солдатских депутатов}. Но подобное двоевластие долго продолжаться не могло. Экономическая политика Временного правительства отвечала интересам буржуазии. Но Временное правительство не смогло справиться с хаосом.

\hfill

В этот критический момент В.И. Ленин выступил с \textbf{апрельскими тезисами}. Большевики во главе с В.И. Лениным не только обосновали переход от буржуазно-демократической революции к социалистической, но и разработали программу переходных мер в области экономики и политики. Эта программа предполагала: \textbf{национализацию всех земель, банковского и страхового дела, крупной промышленности, транспорта, установление рабочего контроля, отмену коммерческой тайны, немедленное прекращение дальнейшего выпуска бумажных денег и преобразование всей налоговой системы, введение всеобщей трудовой повинности}.

\pagebreak
\subsection{Дискуссии о характере экономического развития страны в 1917 году}

\pagebreak
\subsection{Альтернативный выход из политического кризиса лета-осени 1917 года}

\end{document}